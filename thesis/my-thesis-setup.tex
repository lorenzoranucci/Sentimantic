% !TEX root = my-thesis.tex


% **************************************************
% Files' Character Encoding
% **************************************************
\PassOptionsToPackage{utf8}{inputenc}
\usepackage{inputenc}


% **************************************************
% Information and Commands for Reuse
% **************************************************
\newcommand{\thesisTitle}{Costruzione di una base di conoscenza Linked Data con tecniche di Machine Learning}
\newcommand{\thesisName}{Lorenzo Franco Ranucci}
\newcommand{\thesisSubject}{Tesi}
\newcommand{\thesisDate}{19 Aprile 2018}
\newcommand{\thesisVersion}{Versione 1}

\newcommand{\thesisFirstReviewer}{Valentina Poggioni}
\newcommand{\thesisFirstReviewerUniversity}{\protect{Università degli Studi di Perugia}}
\newcommand{\thesisFirstReviewerDepartment}{Dipartimento di Matematica e Informatica}

\newcommand{\thesisSecondReviewer}{}
\newcommand{\thesisSecondReviewerUniversity}{\protect{Clean Thesis Style University}}
\newcommand{\thesisSecondReviewerDepartment}{Department of Clean Thesis Style}

\newcommand{\thesisFirstSupervisor}{}
\newcommand{\thesisSecondSupervisor}{}

\newcommand{\thesisUniversity}{\protect{Università degli Studi di Perugia}}
\newcommand{\thesisUniversityDepartment}{Dipartimento di Matematica e Informatica}
\newcommand{\thesisUniversityInstitute}{}
\newcommand{\thesisUniversityGroup}{}
\newcommand{\thesisUniversityCity}{Perugia PG}
\newcommand{\thesisUniversityStreetAddress}{Via Luigi Vanvitelli, 1}
\newcommand{\thesisUniversityPostalCode}{06123}


% **************************************************
% Debug LaTeX Information
% **************************************************
%\listfiles


% **************************************************
% Load and Configure Packages
% **************************************************
\usepackage[english]{babel} % babel system, adjust the language of the content
\PassOptionsToPackage{% setup clean thesis style
    figuresep=colon,%
    sansserif=false,%
    hangfigurecaption=false,%
    hangsection=true,%
    hangsubsection=true,%
    colorize=full,%
    colortheme=bluemagenta,%
    bibsys=biber,%
    bibfile=bib-refs,%
    bibstyle=alphabetic,%
    wrapfooter=false,%
}{cleanthesis}
\usepackage{cleanthesis}

\hypersetup{% setup the hyperref-package options
    pdftitle={\thesisTitle},    %   - title (PDF meta)
    pdfsubject={\thesisSubject},%   - subject (PDF meta)
    pdfauthor={\thesisName},    %   - author (PDF meta)
    plainpages=false,           %   -
    colorlinks=false,           %   - colorize links?
    pdfborder={0 0 0},          %   -
    breaklinks=true,            %   - allow line break inside links
    bookmarksnumbered=true,     %
    bookmarksopen=true          %
}
